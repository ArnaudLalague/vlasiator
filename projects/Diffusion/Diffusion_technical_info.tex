\documentclass[a4paper,10pt]{scrartcl}
\usepackage[utf8x]{inputenc}
\usepackage[left=1.5cm, right=1.5cm, top=1.5cm, bottom=2.5cm]{geometry}
\usepackage[autolanguage,np]{numprint}
\usepackage{tabularx}

\usepackage{hyperref}
\hyperbaseurl{.}

%opening
\title{
\Huge{Vlasiator test cases technical information} \\
\LARGE{Diffusion}
}
\author{Yann Kempf}
\date{Updated on \today}

\begin{document}

\maketitle

\begin{abstract}
   This document gives technical information on the Diffusion test case.
\end{abstract}

\section{Purpose}
Look at the spatial diffusion of a Gaussian blob of density of amplitude 5 in a constant density background of amplitude 1.

\section{Implementation}
The test case is two-dimensional, the scale width of the density blob in $x$ and $y$ is customisable. A magnetic field along $z$ is possible.

\section{Options}
The options available in the \verb=cfg= file are:

\begin{tabularx}{\textwidth}{lX}
   \verb=B0= & $z$ magnetic field (T) \\
   \verb=rho= & Number density (m\textsuperscript{-3}) \\
   \verb=Temperature= & Temperature (K) \\
   \verb=Scale_[xy]= & Scale width of the Gaussian blob of density (amplitude 5 times \verb=rho=) in $x$/$y$ \\
   \verb=nSpaceSamples= & Number of sampling points along spatial dimensions within a spatial cell, includes the corners (minimum 2) \\
   \verb=nVelocitySamples= & Number of sampling points along velocity dimensions within a velocity cell, includes the corners (minimum 2)
\end{tabularx}




\end{document}