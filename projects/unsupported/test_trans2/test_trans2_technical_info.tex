\documentclass[a4paper,10pt]{scrartcl}
\usepackage[utf8x]{inputenc}
\usepackage[left=1.5cm, right=1.5cm, top=1.5cm, bottom=2.5cm]{geometry}
\usepackage[autolanguage,np]{numprint}
\usepackage{tabularx}

\usepackage{hyperref}
\hyperbaseurl{.}

%opening
\title{
\Huge{Vlasiator test cases technical information} \\
\LARGE{test\_trans2}
}
\author{Yann Kempf}
\date{Updated on \today}

\begin{document}

\maketitle

\begin{abstract}
   This document gives technical information on the test\_trans2 test case.
\end{abstract}

\section{Purpose}
Investigate spatial/velocity resolution issues more deeply after the appearance of \textit{blocky} features in the Diffusion test case.


\section{Implementation}
A ring is initialised with radial velocity in the $xy$-plane, set in periodic boundary conditions. Due to the finite resolution box-cars appear so that the ring is not smooth. The definition of a uniform initial ring with exact radial velocity is tricky, the best is to match the spatial and velocity grids so that cell indices can be matched. This is implemented as well as a velocity condition to select zero $z$ velocity.

\section{Options}
The options available in the \verb=cfg= file are:

\begin{tabularx}{\textwidth}{lX}
   \verb=radLimitInf= & radial inner radius in space \\
   \verb=radLimitSup= & radial outer radius in space
\end{tabularx}




\end{document}