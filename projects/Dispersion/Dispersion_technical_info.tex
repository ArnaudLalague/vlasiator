\documentclass[a4paper,10pt]{scrartcl}
\usepackage[utf8x]{inputenc}
\usepackage[left=1.5cm, right=1.5cm, top=1.5cm, bottom=2.5cm]{geometry}
\usepackage[autolanguage,np]{numprint}
\usepackage{tabularx}

\usepackage{hyperref}
\hyperbaseurl{.}

%opening
\title{
\Huge{Vlasiator test cases technical information} \\
\LARGE{Dispersion}
}
\author{Yann Kempf}
\date{Updated on \today}

\begin{document}

\maketitle

\begin{abstract}
   This document gives technical information on the Dispersion test case.
\end{abstract}

\section{Purpose}
Produce dispersion plots. Runs based on Fluctuations in one dimension, with fully periodic boundary conditions. The tool \verb=vlsv2bzt_[SD]P= extracts the ($x$-$t$)-dataset of any variable, which can then be processed using Scilab or MATLAB to obtain the ($k$-$\omega$) plot. The \verb=Dispersion.m= script does this, using \verb=Hamming.m= for windowing in time if wanted.


\section{Implementation}
This test uses a gas factory-type system to initialise a consecutive block of cells of random length with the same values. The rationale was to lower the chances of nasty shocks by reducing the number of cell-interface discontinuities. It has actually not been of great help and was conflicting with seeding problems so that actually Fluctuations was used in the production runs for my MSc thesis.

\section{Options}
The options available in the \verb=cfg= file are:

\begin{tabularx}{\textwidth}{lX}
   \verb=B[XYZ]0= & Background magnetic field (T) \\
   \verb=rho= & Number density (m\textsuperscript{-3}) \\
   \verb=Temperature= & Temperature (K) \\
   \verb=magPertAmp= & Absolute amplitude of the magnetic field perturbations (T) \\
   \verb=densityPertAmp= & Relative amplitude of the density perturbations \\
   \verb=velocityPertAmp= & Absolute amplitude of the velocity perturbations (m/s) \\
   \verb=seed= & Multiplied by the MPI rank to seed the \verb=rand()= RNG \\
   \verb=sectorSize= & Maximal length of a consecutive sector of constant valued cells \\
   \verb=nSpaceSamples= & Number of sampling points along spatial dimensions within a spatial cell, includes the corners (minimum 2) \\
   \verb=nVelocitySamples= & Number of sampling points along velocity dimensions within a velocity cell, includes the corners (minimum 2)
\end{tabularx}




\end{document}