\documentclass[a4paper,10pt]{scrartcl}
\usepackage[utf8x]{inputenc}
\usepackage[left=1.5cm, right=1.5cm, top=1.5cm, bottom=2.5cm]{geometry}
\usepackage[autolanguage,np]{numprint}
\usepackage{tabularx}

\usepackage{hyperref}
\hyperbaseurl{.}

%opening
\title{
\Huge{Vlasiator test cases technical information} \\
\LARGE{Harris}
}
\author{Yann Kempf}
\date{Updated on \today}

\begin{document}

\maketitle

\begin{abstract}
   This document gives technical information on the Harris test case.
\end{abstract}

\section{Purpose}
Produce an MHD Harris equilibrium. Does not stay in equilibrium most likely due to the lack of the Hall and electron pressure terms in Ohm's law.


\section{Implementation}
Adds a Harris sheet-type profile along $x$ (non-periodic) of amplitude 5 on top of a constant background of amplitude 1. The magnetic field is along $z$, the width of the sheet can be set.

\section{Options}
The options available in the \verb=cfg= file are:

\begin{tabularx}{\textwidth}{lX}
   \verb=Scale_size= & Scale width of the Harris sheet (m) \\
   \verb=B0= & Magnetic field at infinity (T) \\
   \verb=Temperature= & Temperature (K) \\
   \verb=rho= & Number density (m\textsuperscript{-3})
\end{tabularx}




\end{document}